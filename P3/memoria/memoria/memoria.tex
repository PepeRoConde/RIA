\documentclass[12pt,a4paper]{article}

% Paquetes necesarios
\usepackage[utf8]{inputenc}
\usepackage[spanish]{babel}
\usepackage{graphicx}
\usepackage{amsmath}
\usepackage{amsmath,amssymb}  
\usepackage{geometry}
\usepackage{float}
\usepackage{caption}
\usepackage{subcaption}
\usepackage{hyperref}
\usepackage{listings}
\usepackage{xcolor}

% Configuración de página
\geometry{a4paper, margin=2.5cm}
\setlength{\parindent}{0pt}
\setlength{\parskip}{6pt}

% Configuración de código
\lstset{
    basicstyle=\small\ttfamily,
    breaklines=true,
    frame=single,
    language=Python
}

% Título y autores
\title{\textbf{Práctica 3: Detección en tiempo real}}
\author{Marcelo Ferreiro Sánchez \\ Pepe Romero Conde}
\date{\today}

\begin{document}

\maketitle

\section{Práctica 3.1: Seguimiento de Objeto Estático}

\subsection{Definición del Problema}

Esta práctica que se presenta en dos partes, la primera consiste de la
implementación de un sistema de detección de posición corporal mediante el cual
podamos controlar al robot robobo, mientras que el objetivo de la siguiente es
dotarlo de un sistema de detección de objetos gracias al cual poder adaptar la
política de la Práctica 1 para que, al encontrar un objeto objetivo, se active
la misma y se acerque activamente al mismo.


\subsection{Estructura de la solución propuesta}
\begin{figure}[H]
    \centering
    \includegraphics[width=1.3\linewidth]{../diagramas/flujo.drawio_2.png}
    \caption{Diagrama de flujo del sistema. Para cada \texttt{step()}, la \texttt{observación} del mundo real (conformada por el estado interno del Robot y de lo que ve por la cámara) se usa para decidir que subsistema tomará la siguiente acción. 
    Si no se ve al objeto, tomará el control el \color{orange}{modelo telecontrol}, si no la \color{blue}{política adaptada de la Práctica 1}.}
    \label{fig:diagrama}
\end{figure}

En esta práctica, para mantener la coherencia con la P1, decidimos continuar
utilizando el entorno definido en la misma (Clase entorno que hereda de la clase
\texttt{gym.Env}) tanto para la ejecución de la política como para el movimiento
por telecontrol.

Para gestionar el cuándo y cómo cambiar del modelo de telecontrol se creó una
clase Modelo con un método \texttt{predict()}, el cual, en caso de estar viendo
el objeto objetivo, llamará a la política de la P1 y en caso contrario manejará
el robobo por telecontrol (YOLO - pose). La lógica para saber si el robot ve el objeto
objetivo se maneja gracias a la función \texttt{esta\_viendo()}, que llamará al
YOLO de detección de objetos para averiguar si existe una instancia del objetivo.






\subsection{Adaptaciones a entorno real}


Hubieron de hacerse algunos cambios a la representación de las acciones y del
espacio de observaciones del entorno, tales como, por parte de las acciones del
telecontrol, transformarlas al formato en que el entorno las representa y, por
parte de la detección de objetos, se tuvo que buscar formas de representar las
coordenadas en pantalla del objetivo y su tamaño. 

La forma de conseguir las coordenadas en pantalla, fue la del calcular el centro
del rectángulo en que YOLO encuadra el objeto, y normalizar esa posición a
valores en [0,100] para el valor de cada eje. De forma similar, el tamaño se
calculó como el área del mismo rectángulo.

La noción de 3D para Robobo se ve supeditada por el
\texttt{Enorno.observacion[`tamano\_blob']}, es lo unico que tiene para saber si
esta cerca o lejos. Por eso hemos tenido que diseñar el sistema de una forma
modular y con un archivo \texttt{config.yaml} para la fácil configuración de
estos (muchos) hiperparámetros . También cuestiones relacionadas con la
brusquedad del giro han necesitado ser adaptadas. Todos estos cambios se
hicieron com miras a conseguir un correcto desempeño del robot en el mundo real,
haciéndolo parecerse al máximo a su comportamiento en simulador.



\subsection{Resultados y conclusiones}
Realizando las pruebas finales en el mundo real, se pudo comprobar como se
consiguió un sistema de telecontrol adecuado, si bien con algo de latencia
debido al tiempo que consume el sistema YOLO en detectar los puntos clave del
cuerpo del usuario y la latencia inherente a la comunicación inalámbrica.

En cuanto al comportamiento de la detección de objetos y activación de la
política, el robot la activa con bastante rapidez y es capaz de acercarse al
objeto. También, en caso de perder al mismo, el sistema de telecontrol vuelve a
ser el vigente rápidamente, permitiendo así salvar algunas situaciones en las
que el YOLO puede fallar, prediciendo que no hay botella durante la ejecución de
la política.





\end{document}